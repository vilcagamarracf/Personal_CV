\section{Experiencia laboral }
\vspace{4px}
% ----------------------------------
\cventry{2022-2023}{Asistente de investigación}{}{\httplink[Proyecto RICEMON (UPV-UNALM)]{https://ricemon.webs.upv.es/}}{}{}

\cvline{}{Actividades realizadas en Ferreñafe, Lambayeque (1 mes)}
\cvlistitem{Manejo de instrumentos}
\cvlistitem{Recolección de datos de campo (NDVI y LAI)}
\cvlistitem{Fotografía y filmación de talleres realizados por el proyecto.}

\vspace{2px}

\cvline{}{Actividades realizadas en Lima}
\cvlistitem{Procesamiento de datos obtenidos en Ferreñafe, Lambayeque}
\cvlistitem{Elaboración de modelo de estimación de evapotranspiración con sensores remotos.}
\cvlistitem{Desarrollo de tesis (en curso) con nombre \textit{"Aplicación de Google Earth Engine y el modelo METRIC en la estimación de evapotranspiración para campos de arroz: Caso de estudio en Lambayeque, Perú"}.}

\vspace{5px}
% ----------------------------------
\cventry{2020}{Asistente de gabinete}{}{3 Meses (Marzo - Mayo)}{}{}

\cvline{}{Prácticas pre-profesionales en Huancayo, Junín. }
\cvlistitem{Elaboración de planos en AutoCAD para pequeños proyectos de implementación y mantenimiento de sistemas de riego en la región de Churcampa, Huancavelica.}

\vspace{5px}
% ----------------------------------
\cventry{2019}{Practicante}{}{1 Mes (Julio - Agosto)}{}{}

\cvline{}{Prácticas pre-profesionales en Lima}
\cvlistitem{Extracción de datos PISCO con lenguaje de programación R}
\cvlistitem{Elaboración de mapas temáticos}

% ----------------------------------
\section{Producción científica }
\vspace{4px}

\cvline{2023}{Redacción (en curso) del artículo \textit{"Application of Google Earth Engine and the METRIC model in estimating evapotranspiration for paddy: case study in Lambayeque, Perú"} sobre campaña agrícola 2022 con el proyecto RICEMON (UPV-UNALM).}